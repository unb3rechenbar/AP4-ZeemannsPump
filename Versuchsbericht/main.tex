\documentclass[
    a4paper,
    twoside=true,
    footinclude=off, 
    captions=tableheading, 
    DIV=12;usenames,
    dvipsnames
]{scrartcl}

% -----------------------------------------------------------
% Pakete
\usepackage{UniLaTeXPackage}

\begin{document}

% -----------------------------------------------------------
% Deckblatteinrichtung
    \title{Optisches Pumpen und Zeemann-Effekt}
    \subtitle{(Anfänger/Fortgeschrittenen) Praktikum i in Physik (Universität Konstanz)}
    \author{Authoren: Tom Folgmann, David Jannack \\ \large{Tutor: Maximilian Mattes}}
    \date{Experimente durchgeführt am 20.06.2023}
    \maketitle
    \pagenumbering{alph}
    \thispagestyle{empty}
    \section*{Einleitung}
        \subfile{Sektionen/0-Einleitung.tex}

    \newpage

% -----------------------------------------------------------
% Inhaltsvorbereitung

    \tableofcontents
    \thispagestyle{empty}	
    \newpage
    \setcounter{page}{1}
    \pagenumbering{arabic}

% -----------------------------------------------------------
% Dokumentkern

\newpage
\section{Grundlagen}
    \subfile{Sektionen/1-Grundlagenteil.tex}
    \subfile{Sektionen/2-UnbeantworteteFragenUndAufgaben.tex}
	

\newpage
\section{Versuchsdurchführung}
    \subfile{Sektionen/3-Versuchsdurchfuehrung.tex}

\newpage
\section{Auswertung}
    \subfile{Sektionen/4-Auswertung.tex}

\newpage
\section{Fazit}
    \subfile{Sektionen/5-Fazit.tex}

% -----------------------------------------------------------
% Abspann und Credits

\newpage
    \subfile{Sektionen/6-Literatur.tex}
\newpage
    \listoffigures
    \listoftables


\newpage
    \section*{Messdaten}
    \subfile{Sektionen/7-Messdaten.tex}
\newpage
    \subfile{Sektionen/8-Auswertungscode.tex}



%\includepdf[pages=1-1,scale=0.9,frame=true]{Protokolle/PATH.pdf}
\end{document}
