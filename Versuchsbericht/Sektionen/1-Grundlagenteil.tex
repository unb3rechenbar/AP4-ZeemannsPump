\documentclass[../main.tex]{subfiles}
\begin{document}
    \subsection{Eigenschaften des Spektrums}
        \subsubsection*{Notation}
            % -> Erkläre die Spektroskopische Notation
            Bei der \emph{spektroskopischen} Notation handelt sich um eine Formalität zur Darstellung der Energieniveaus eines Atoms. Man überführt dabei das Tupel $(s,j,l)$ in die Schreibweise $^{2\cdot s + 1}(\mcB(l))_j$, wobei $\mcB$ eine Buchstabenzuordnung der Form $0\mapsto S$, $1\mapsto P$, $2\mapsto D$ usw. ist. 

        \subsubsection*{Zeeman-Effekt}
            % -> Standpunkt der klassischen Elektronentheorie
            % -> Polarisationseffekte
            % -> Schaubild

        \subsubsection*{Fein- und Hyperfeinstrukturen}

        \subsubsection*{Anomaler Zeeman-Effekt}
            % -> Aufspaltung von zwei Energieniveaus und zug. Spektrallinien des Überangangs zeigen (Beispiel)
            % -> Unterschied zum normalen Zeeman-Effekt
            % -> Unterschiedlich starke Aufspaltung der Energieniveaus per Vektormodell
            % -> Landé-Faktoren (auch beim reinen Spin- und reinem Bahnmagnetismus)

    \subsection{Rubidium 85 und 87}
        % -> Unterschiede, Auswirkungen auf die Zeeman-Aufspaltung
        % -> Aufspaltung des Energieniveaus durch Zeeman-Effekt


    \subsection{Technisches}
        \subsubsection*{Vielstrahlinterferometer}
            
        \subsubsection*{Justierung}


\end{document}