\documentclass[../main.tex]{subfiles}
\begin{document}
    \subsection{Eigenschaften des Spektrums}
        \subsubsection*{Notation}
            % -> Erkläre die Spektroskopische Notation

        \subsubsection*{Zeeman-Effekt}
            % -> Standpunkt der klassischen Elektronentheorie
            % -> Polarisationseffekte
            % -> Schaubild

        \subsubsection*{Fein- und Hyperfeinstrukturen}

        \subsubsection*{Anomaler Zeeman-Effekt}
            % -> Aufspaltung von zwei Energieniveaus und zug. Spektrallinien des Überangangs zeigen (Beispiel)
            % -> Unterschied zum normalen Zeeman-Effekt
            % -> Unterschiedlich starke Aufspaltung der Energieniveaus per Vektormodell
            % -> Landé-Faktoren (auch beim reinen Spin- und reinem Bahnmagnetismus)

    \subsection{Rubidium 85 und 87}
        % -> Unterschiede, Auswirkungen auf die Zeeman-Aufspaltung
        % -> Aufspaltung des Energieniveaus durch Zeeman-Effekt


    \subsection{Technisches}
        \subsubsection*{Vielstrahlinterferometer}

        \subsubsection*{Justierung}
        

\end{document}