\documentclass[../main.tex]{subfiles}

\begin{document}
    
\subsection{Optisches Pumpen}

\subsection{Zeeman-Effekt}

Tabelle \ref{tab:BFeldZeeman} zeigt die Messwerte der magnetfischen Flussdichte, bei der die normale Zeeman-Aufspaltung auftritt.

\begin{table}[H]
    \centering
    \begin{tabular}{l|l|l}
        Messung & Transversales Feld in $T$ & Longitudiales Feld in $T$\\
        \hline\hline
        n=1 & \num{0.600(10)} & \num{0.457(25)}\\
        \hline
        n=2 & \num{0.560(10)} & \num{0.480(25)}\\
        \hline
        n=3 & \num{0.570(10)} & \num{0.497(25)}\\
        \hline
        n=4 & \num{0.570(10)} & \num{0.468(25)}\\
        \hline
        n=5 & \num{0.580(10)} & \num{0.460(25)}\\
        \hline\hline
        Mittelwerte & \num{} & \num{}
    \end{tabular}
    \caption{B-Feld für Zeeman-Aufspaltung bei einer Cadmium-Lampe}
    \label{tab:BFeldZeeman}
\end{table}

\noindent Desweiteren zeigen die aufgespalteten Spektrallinien bei longitudialem und transversalem ein anderes Polarisationsverhalten. Ergernererneneneneneneennenenenenenenenene


\end{document}