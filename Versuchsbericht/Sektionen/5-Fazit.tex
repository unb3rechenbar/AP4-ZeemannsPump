\documentclass[../main.tex]{subfiles}
\begin{document}

    Die mittels der Zeeman-Aufspaltung erhaltenen Werte für die spezifische Elementarladung betragen: $(e/m)_t=\SI{1.870(15)e11}{\coulomb\per\kilo\gram}$ (bei paralleler Ausrichtung von $B$-Feld und Strahlung) und $(e/m)_l=\SI{1.710(17)e11}{\coulomb\per\kilo\gram}$ (bei senkrechter Ausrichtung von $B$-Feld und Strahlung). Diese Ergebnisse sind sigifikant diskrepant zum Literaturwert, unterscheiden sich aber dennoch um nur ca. $\SI{10}{\percent}$. Für die parallele Lage des Magnetfeld ergaben sich weiter eine zirkular polarisierte Emisssionsline und zwei lineare polarisierte. Für die senkrechte Lage des Feldes sind diese Zahlen genau umgekehrt.\\
    
    \noindent Durch das optische Pumpen an zwei Rubidium-Isotopen $^{87}$Rb und $^{85}$Rb konnten die Kernspinzahlen $I_1=1.478(20)\approx 1.5$ und $I_2=3.417(20)\approx 3.5$ bestimmt werden. $I_1$ ist in guter Übereinstimmung mit den zu erwartenden Wert $3/2$ für $^{87}$Rb, $I_2$ ist allerdings weitaus größer als der theoretische Wert von $5/2$ für $^{85}$Rb.
    

\end{document}