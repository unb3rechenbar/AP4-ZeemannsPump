\documentclass[../1-Grundlagenteil.tex]{subfiles}
\begin{document}

\subsubsection*{Termschema von $^{87}$Rb}
    Rubidium $^{87}$Rb hat Zustände mit eine Hauptquantenzahl $n\in\{1,2,...,5\}$, wobei für $n=5$ nur ein Elektron vorhanden. Für Wechselwirkung mit diesen Schalenelektron ist der Gesamtspin also $S=1/2$. Grafik \ref{fig:Rb87Termschema} zeigt die ersten drei Energieniveaus für $n=5$.

    \begin{figure}[H]
        \centering
        \includegraphics[width=8cm]{../../Bilddateien/Grundlagen/Rb87Termschema.jpg}
        \caption{Energieniveaus $^2\text{S}_{1/2}$, $^2\text{P}_{1/2}$ und $^2\text{P}_{3/2}$ von $^{87}$Rb und deren (Hyper-)Feinstrukturaufspaltung, sowie deren Zeeman-Aufspaltung \cite[p.171]{Schneider}}
        \label{fig:Rb87Termschema}
    \end{figure}

    Für diesen Versuch werden nur Übergänge zwischen Niveaus im Grundzustand betrachtet, für die $L=0$ gilt. Der Gesamtdrehimpuls ist also $J=L+S=S=1/2$.

\subsubsection*{Optisches Pumpen}
    
    
\subsubsection*{Unterschiede zwischen $^{87}$Rb und $^{85}$Rb}
    Das Isotop $^{87}$Rb weißt den Kernspin $I_{87}=\frac{3}{2}$ auf, und $^{85}$Rb den Spin $I_{85}=\frac{5}{2}$. Diese unterschiedliche Kern-Bahn-Wechselwirkung sorgt zum einen für eine unterschiedlich starke Zeeman-Aufspaltung. Zum anderen sorgt die verschiedene Anzahl vorhandener Hyperfeinniveaus auch für eine verschiedene Anzahl von Zeeman-Niveaus \cite[180]{Schneider}.

\end{document}