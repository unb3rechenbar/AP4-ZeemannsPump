\documentclass[../main.tex]{subfiles}
\begin{document}

Die Zeeman-Aufspaltung des Spektrums von Atomen kann zu genauen Messungen atomarer Größen genutzt werden. Hier wird examplarisch die spezifische Elementarladung $e/m_e$ anhand dieser Aufspaltung bestimmt werden. Desweiteren werden die beim normalen Zeeman-Effekt emittierten Spektrallinien auf ihre Polarisation hin untersucht.\\

Eine weitere Möglichkeit, die Zeeman-Aufspaltung zu untersuchen, ist das optische Pumpen. Im Rahmen des Versuchs werden die Grundzustand-Übergänge in zwei Rubidium-Isotopen beobachtete und daraus die Kernspinzahl des jeweiligen Isotops berechnet.

%\begin{figure}[H]
%    \centering
%    \includegraphics[width=10cm]{Bilddateien/CoffinDance.jpg}
%    \label{fig:myfreshbild}
%\end{figure}

\end{document}